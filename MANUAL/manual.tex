\documentclass[10pt,a4paper]{article}
\usepackage{geometry}
 \geometry{
 a4paper,
 total={170mm,257mm},
 left=20mm,
 top=20mm,
 }
\setlength{\parindent}{0pt}
\setcounter{secnumdepth}{0} % sections are level 1
\usepackage[utf8]{inputenc}
\usepackage{amsmath}
\usepackage{amsfonts}
\usepackage{amssymb}
\usepackage{fancyvrb}
\usepackage{hyperref}
\usepackage{titlepic}
\usepackage{graphicx}
\usepackage{eso-pic}

\DefineVerbatimEnvironment{verbatim}{Verbatim}{xleftmargin=.5in}
\hypersetup{
    colorlinks = true,
    allcolors = {blue},
}

\titlepic{\includegraphics[width=0.5\textwidth]{logo.png}}
\title{\textbf{FracG V 0.1} - \\
Graph based Fault and Fracture Analysis}
\date{2019\\ April}
\author{by some guys...  \\
{\small CSIRO Deep Earth Imaging }\\
{\small Future Science Platform}}
\begin{document}
\maketitle
\newpage
\tableofcontents
\vspace*{\fill}
FracG is free software: you can redistribute it and/or modify
it under the terms of the GNU General Public License as published by
the Free Software Foundation, either version 3 of the License, or
(at your option) any later version.
FracG is distributed in the hope that it will be useful,
but WITHOUT ANY WARRANTY; without even the implied warranty of
MERCHANTABILITY or FITNESS FOR A PARTICULAR PURPOSE.  
See the GNU General Public License for more details.
You should have received a copy of the GNU General Public License along with FracG.  
If not, see \url{https://www.gnu.org/licenses/}.

\newpage
\section{Installation}
\textit{FracG} is designed for Debian GNU/Linux and depend on three main libraries: GDAL~\cite{gdal18}, ARMADILLO~\cite{san16} and BOOST~\cite{sie02, geh16}.

First, get the latest GDAL/OGR version, add the PPA to your sources:
\begin{verbatim}
	sudo add-apt-repository ppa:ubuntugis/ppa && sudo apt-get update
\end{verbatim}

Now the necessary libraries can be installed from the terminal:
\begin{verbatim}	
	sudo apt-get install build-essential \
	libboost-all-dev \
	liblapack-dev \
	libblas-dev \ 
	libboost-dev \
	libgsl-dev \
	libarmadillo-dev 
\end{verbatim}

Export the environmental variables for gdal 
\begin{verbatim}	
	export CPLUS_INCLUDE_PATH=/usr/include/gdal
	export C_INCLUDE_PATH=/usr/include/gdal
\end{verbatim}

To obtain \textit{gmsh} visit \url{http://gmsh.info/}. Note that open cascade is used for creating the mesh's and obtaining the pre-compiled version is recommended.

Two options are available for automatic compilation:

\subsection{xutils}
This options uses a shell script and and Imake file.
\begin{verbatim}
	sudo apt-get install xutils-dev 
\end{verbatim}

If the libraries are all installed correctly the permission for the file ``\textit{install.sh}'' need to be changed.
\begin{verbatim}
	cd ...FracG
	sudo chmod -x install.sh 
\end{verbatim}
Now the installation script can be executed:
\begin{verbatim}
	sudo ./install.sh
\end{verbatim}
\textit{FracG} can now be executed from the command line.

\subsection{cmake}
This option is recommended and used CmakeList.txt.
\begin{verbatim}
	sudo apt-get install cmake
\end{verbatim}

In the \textit{FracG} directory, type;
\begin{verbatim}
	mkdir build \
	cd build \
	cmake .. \
	sudo make install
\end{verbatim}
\textit{FracG} can now be executed from the command line.

\subsection{manual compilation}
You can also compile \textit{FracG} from a terminal without the install script. Please note that this will make \textit{FracG} only executable from within the directory you compiled it in.

\begin{verbatim}
	cd ...FracG/src
	g++ -o FracG main.cpp graph.cpp GeoRef.cpp geometrie.cpp stats.cpp 
	-larmadillo -lgsl -lgdal -lgmsh 
\end{verbatim}

\newpage


\section{Setting up the analysis}


\section{Example: Test1}


\section{Classes and functions}

\subsection{Typedefinitions FGraph}

\begin{verbatim}
point_type;
\end{verbatim}

\begin{verbatim}
RASTER;
\end{verbatim}

\begin{verbatim}
typedef geometry::model::linestring<point_type> line_type;
\end{verbatim}





\subsection{CLASS GeoRef}

\subsection{CLASS GEOMETRIE}

\subsection{CLASS GRAPH}

\subsection{CLASS STATS}
\subsubsection{AnalyseRaster}
This function calculates the statistics of the network within the are of a specified raster file.
The raster data at lineaments centres adn th gradient across the centres are obtained.
Graph's algorithms shortest path and maximum flow are performed utilizing the data of the raster file.
The outputs two CSV files containing the geometrical statistics fro the initial lineament set that is contained in the raster file, the geometrical statistics fro the graph and two ESRi shapefiles, one fro the vertices and one for the edges of the graph. 
\begin{verbatim}
void AddData(string name, vector<line_type> lines, point_type source, 
point_type target, double Dist)
\end{verbatim}
-\textbf{name:} name of or path to the raster-file \\
-\textbf{lines:} vector containing the line-strings of the lineaments \\
-\textbf{source:} coordinates of the point used as source for the shortest path and the maximum flow algorithm \\
-\textbf{target:} coordinates of the point used as target for the shortest path and maximum flow algorithm \\
-\textbf{return type}: void \\
-outputs: 

\subsubsection{AnalyseRasterLines}
\begin{verbatim}
 void AnalyseRasterFaults(vector<READ> input, vector<line_type> faults)
\end{verbatim}




\section{CLASS MODEL}


\begin{table}[h!]
  \begin{center}
    \caption{Output files.}
    \label{tab:table1}
    \begin{tabular}{l|l} % <-- Alignments: 1st column left, 2nd middle and 3rd right, with vertical lines in between
      \textbf{Filename} & \textbf{Content} \\
      \hline
      Initial\_Fault\_Statistics.txt & Gemetric properties, CDF of length, k-means of geometries \\
      BoxCounting.txt & Number of non-empty boxes, box-size and estimated Minkowski dimension \\
      Graph\_results.txt &  Information on topology derived from graph analysis\\
      GRAPH.shp & Shapefile containing the branches of the graph with associated attributes  \\
      \textit{vec\_filename}\_results.txt & Location, degree and elevation of graph vertices \\
      DEM-\textit{ras-filename}\_vertexData.txt & type component and elevation values around graph vertices \\
      DEM-\textit{ras-filename}\_EdgeData.txt & edge number, edge type, component,elevation data and slope\\
      MinTree.shp & Shapefile containing the Kruskal Minimum Spanning tree \\
      ShortPath.shp & Shapefile contain the Dijkstra shortest distance between two points \\
      FaultData\_\textit{ras-filename}.txt & Data around fault traces from additional raster file \\
    \end{tabular}
  \end{center}
\end{table}



\section{Trouble shooting}
In some cases it might be necessary to install Aramdillo manually. In that case please follow the instruction on the web page:
~\href{http://arma.sourceforge.net/download.html}{ARMADILLO}

If you should encounter any difficulties with compiling \textit{FracG} please~\href{mailto:uli.kelka@csiro.au}{email} us a description of the problem.

\vspace*{\fill}
\bibliography{manual_bib} 
\bibliographystyle{ieeetr}
\end{document}